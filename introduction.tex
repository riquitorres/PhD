\chapter{Introduction}
\section{Eastern Boundary Systems}
This thesis will investigate the seasonality of the Galician shelf
(North-West of Spain). The region forms part of the wider East
Atlantic Boundary System, which includes the Portugal and Canary
Island systems. Like most eastern regions, Galicia experiences a
marked seasonality mainly driven by the wind regime, but also by
an oceanic density gradient resulting in summer upwelling and
winter slope poleward flow. The Galician shelf is the northernmost
limit of the Atlantic Eastern Boundary upwelling system and is
characterised by the abrupt change in coast orientation at Cape
Finisterre, where the coast changes from a N-S to E-W direction.
The relatively narrow shelf is populated by ridges and canyons.
The coastline also has abrupt indentations, the ``Rias'', which
together with the Minho River, provide the freshwater input onto
the shelf. The complexity of the coastline directly affects the
wind inducing spatial and temporal variability which transfers to
the shelf. During the upwelling regime, the region is also the
site of at least 3 filaments while during the downwelling regime,
eddies are shed from the poleward flow, preferentially at capes.
It all combines to produce a highly complex and dynamic system.

There is a basic lack of understanding of the Galician shelf
circulation and its relation to the wind forcing. The Ekman index
has been used as a first approximation for the shelf circulation
and dynamic state, but the complexity of the coastline makes the
assumption of single point geostrophic wind measurements
representative of the wider shelf region difficult to hold. The
area has a very high marine traffic of heavy ships/containers and
constitutes one of the main marine corridors in Europe. Accidents
like the recent and sadly famous oil spill from the ``Prestige''
are not uncommon. In the last 30 years Galicia has been the site
of 5 more oil spills (``Polycommander'', ``Urquiola'',
``Erkowit'', ``Andros Patria'', ``Cas\'on'' and ``Mar Exeo'')
making it the world's oil spill hot spot. To this day, detailed
information of the shelf oceanography is still lacking. Marine
traffic through the harbour of Vigo, in the southernmost Ria, is
also high but very little is known that could help in case of
emergency. While Galicia constitutes one of the least productive
Eastern upwelling areas, upwelling has an important effect inside
the ``Rias''. The ``Rias Bajas'' (the three southernmost ``Rias'')
represent one of the largest mussel production regions in the
world and much of the recent research has focused on the
circulation and biochemical fluxes inside the Rias. Some of this
research has shown there is a strong link between the Rias and the
shelf, e.g. upwelling takes place inside the Rias in response to
equatorward winds. Shelf circulation features directly influence
the Rias, e.g. nearshore poleward coastal current has been related
to the generation of harmful algal blooms, and the shelf
circulation needs to be characterized in order to further our
knowledge of the whole system.

\section{Aims of the thesis}
This thesis will attempt to describe the main features of the
Galician region at different stages of its seasonal cycle. Three
main data groups have been collected for that purpose:

\begin{description}
  \item[Cruise data] Data from 4 cruises during the
  Spring spin up, Summer upwelling, Autumn transition and
  Winter downwelling will be used. They include meteorological
  data, hydrographic data (CTD, Conductivity, Temperature and
  Depth), current data (ADCP, Acoustic Doppler Current Profiler)
  and turbulence data (FLY, Fast, Light, Yo-Yo profiler).
  \item[Remotely sensed data] Data from advanced very high
  resolution radiometer (AVHRR) and Scatterometer wind data have
  been used extensively. Eight mixed layer drifters were released in
  the region and tracked through the ARGOS system.
  \item[Long-term measurements] Include data from meteorological coastal
  stations and shelf moored oceanographic buoys when possible.
  They included hourly winds and surface currents.
\end{description}

The main objective of the thesis is to identify and describe the
key oceanographic features of the Galician coast at the different
seasonal stages. To do so, the aims of the research were to:

- Identify typical wind patterns and characterise the wind
variability.

- Describe the transitional periods. Identify their driving
mechanisms. What happens to the poleward flow in the face of
upwelling winds?. How does the transition between upwelling to
downwelling and vice versa take place?.

- Describe the 3-D structure of a filament in the Galician region
and establish its possible role in the exchanges between shelf and
ocean.

- Obtain a picture of the seasonal differences in the mesoscale
circulation and horizontal diffusion through Lagrangian drifter
releases.

\section{Thesis structure}
This thesis describes the work and results from 4
multidisciplinary oceanographic cruises in the Galician region at
different stages of the general seasonal regimes. It has been
divided into eight Chapters.

\begin{description}
  \item[Chapter~\ref{ch:litrev}] A general description of the hydrography and
  circulation of the region at the seasonal scale is given and
  the main mesoscale structures present are identified. An
  introduction to the physical theory behind upwelling and downwelling
  dynamics is given.
  \item[Chapter~\ref{ch:winds}] Spatial distribution of winds around the
  Galician region is investigated with the first two years of
  daily QuikScat scatterometer winds. Typical wind spatial
  patterns are sought for the different wind regimes.
  \item[Chapter~\ref{ch:spring}] The response of the system at the
  beginning of the upwelling season to sudden changes between
  downwelling and upwelling favourable winds is described on the
  basis of results from cruise CD105.
  \item[Chapter~\ref{ch:summer}] Data from cruise CD114 is used to provide
  a description of filaments during a mature
  stage of the upwelling regime. A study of the effect of
  upwelling relaxation on the upwelling system both over the shelf
  and in filaments is made.
  \item[Chapter~\ref{ch:winter}] The onset of the downwelling
  regime is described. Hydrography and velocity data from cruise THALASSA1099 show the structure of the
  poleward slope flow along the Galician shelf.
  \item[Chapter~\ref{ch:drifters}] The behaviour of Lagrangian surface
  drifters launched during cruises CD114 and METEOR 46/2
  during both upwelling and downwelling regimes is described and
  estimates of horizontal mixing coefficients are made.
  \item[Chapter~\ref{ch:conclusions}] A discussion of all the observations integrates
  results from all chapters to provide a picture of the
  circulation patterns of the Galician system and their seasonal variability.

\end{description}
