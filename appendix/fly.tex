%\coversheet{APPENDIX}
% \notchapter{odd}{APPENDIX}{y}
% \renewcommand{\thechapter}{A}
% \setcounter{section}{0}
\chapter{Supplementary Information to Chapter \ref{ch:summer}}
\section{Processing of FLY data}\label{ap:fly}


A more thorough description of the processing of the FLY data can
be found in \citet{Inall98}.

The free falling FLY (Fast, Light, Yo-Yo) IV profiler was built by
Chris Mackay of Systech Instruments. The instrument is equipped by
two fast response airfoil shear sensors made of a small
piezoceramic bimorph plate that responds to shear strain by
generating a voltage. The sensor is able to detect shears between
0 and 4\shear with a precision of $\pm$5\% and a response length
of 0.01-0.02m \citep{Dewey87} measuring at 280Hz. Two shear
sensors operate in the instrument giving one replica of energy
dissipation.

Further to the shear, the instrument also measures differential
temperature at 140Hz, and conductivity, temperature, tilt, and
pressure at 20Hz, the so called slow response sensors.

The probe is operated by dropping it from the stern of a slowly
moving ship ($\sim$0.25\vel) and letting it free fall, the data
relayed on board through a Kevlar Multi-Conductor cable. While it
falls with a constant velocity $W$ (normally attained within
10-20m of the surface), it experiences a sideways lift force, $F$,
due to the horizontal component of the turbulent velocity
fluctuations, $u$, which is given by \citep{Crawford76}
\begin{equation}
F=\frac{1}{2}\rho\tilde{W}^2A sin 2\phi
\end{equation}

where $\tilde{W}^2=W^2 + u^2$ is the apparent velocity past the
probe; $A$ is the effective cross-sectional area of the probe;
$\rho$ is the density of seawater; and $\phi=tan^-1 \frac{u}{W}$
is the angle of $\tilde{W}$ to the probe. With small $\phi$
($<$5\deg), the instantaneous shear is given by
\begin{equation}
\frac{\partial u}{\partial z}=\frac{1}{CW^2}\frac{\partial
\varsigma}{\partial t}
\end{equation}

where C is a calibration constant and $\varsigma$ (proportional to
$F$) is the output voltage from the sensor. To arrive at the shear
estimates, the fall velocity needs to be generated at 240Hz, which
is done by fitting a third order polynomial to the pressure record
and calculating its first derivative.

The dissipation rate per unit volume, $\epsilon$, is calculated
from the variance of the shear time series
\begin{equation}
\epsilon = 7.5 \nu \overline{\left( \frac{\partial u}{\partial
z}\right) }^2
\end{equation}
where $\nu = 1.049\times 10^6$\mix is the kinematic viscosity of
seawater.

The variance is obtained by integrating the power spectrum of the
shear time series. Subset of the time series of size N=1024 were
used in the main water column resulting in $\sim$2m blocks, while
near the bed, smaller segments were used. The segments overlapped
by N/2 and each were detrended, multiplied by an N point Hanning
cosine window and its power spectral density calculated through
the Welch's method as
\begin{equation}
P(f)=\frac{2\Delta t X(f)^2 V}{N}
\end{equation}

where $2\Delta t$ is the inverse Nyquist frequency, $V$ the
variance lost by applying a cosine window to the series subset,
and $X(f)$ the first N/2 values of the N point Fast Fourier
Transform of the subset. Further corrections are then applied to
account for the decreased in the probe sensitivity to frequencies
higher than 45Hz \citep{Inall98}. Finally, $\epsilon$ is obtained
by integrating $P(f)$ between 1.5Hz and 55Hz.

The conductivity and temperature sensors were compared and
calibrated against simultaneous CTD casts during the CD114 cruise.
A linear correction was found and applied to the conductivity data
as
\begin{equation}
CTDc=1.127\times FLYc -6.4595 (R^2=0.987)
\end{equation}
where CTDc is the conductivity from the CTD and FLYc the
conductivity from the FLY instrument. The temperature data were
found to be in agreement to the CTD data and no correction was
applied.
